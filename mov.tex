\documentclass{article}
\usepackage[utf8]{inputenc}
\usepackage{geometry}
\geometry{
 a4paper,
 total={170mm,257mm},
 left=20mm,
 top=20mm,
 }


\title{Movie review}
\author{Akash, Ankit, Prannay}
\date{April 2017}

\begin{document}

\maketitle

\section*{Rang De Basanti}
A fantastic movie of Mr Perfectionist.

\section*{Swades}

Thank god for swades. Its been ever so often that as Indians we have lamblasted the mushy, one track nonsensical movies that were being churned out by bollywood and yearned for something slightly honest and celebral. And now it seems that the moment has arrived. Swades has finally reached the next level of Indian movie-making with the help of ashutosh, srk and rahman where the execution, the concept and the treatment of the film is at least (i can safely assume) close to world standards. There is no point comparing lagaan (the directors 1st movie) to this movie as his earlier venture was purely fiction with no semblance of realism in it. However this movie has so many ingredients of being a pathbreaker for Indian cinema that its almost scary. The feelings and expectations of an nri set aside, it tells the story of middle and upper class India losing touch with the harsh realities of the sufferings of the less fortunate ones all around us (and the fact that we can do very little to change the ongoing scenario is even more glaring and painful). Besides this the movie also deals with the desires and priorities of humans inspite of the trials and tribulations of daily living.

The maturity of the script, the actors and the score makes it a journey of epic proportions without it looking epic at all. The normalcy of the movie is something which might (and hopefully will) change the way cinema is seen and made in India in the future. One can only hope.



\end{document}

\documentclass{article}
\usepackage[utf8]{inputenc}
\usepackage{geometry}
\geometry{
 a4paper,
 total={170mm,257mm},
 left=20mm,
 top=20mm,
 }


\title{Movie review}
\author{Akash, Ankit, Prannay}
\date{April 2017}

\begin{document}

\maketitle

\section*{Rang De Basanti}
A fantastic movie of Mr Perfectionist.
The director of the movie is Rakesh Om Prakash Mehra.
Paint It Yellow…. Oops….that's suppose to be Rang De Basanti….. Well that's exactly the director tries to convey. Its about today, us and our present, yet the similarities we have from the Pre-independence era. The Gen-x who knows Mac-D but still prefers the Dhaba Paranthas with sweet Lassi. However they restrict their national values only to food and nothing more. Its not a run of the mill stuff with 6 six romantic songs, couple of foreign locales and then finally some emotional drama….. No no no….. Rang De…is a Cult movie. It is more of an introspection, a food for our thought process. It makes us think, as to how should we actually celebrate our freedom. I was really moved by some of the ending lines by the narrator, "I thought there are 2 types of people in the world, one who die crying and other who move away in silence but today I learned there is third genre, people who go laughing". That says it all… 

Rang De… is definitely a very brave and innovative attempt by the director and for that matter he has selected a near perfect cast. However veterans like Om Puri and Anupam Kher looked disposed. All the characters grow gradually in the film and make you think their way. Everyone is given enough space to justify their talent. Rakesh Omprakash Mehra is a director of the new emerging Indian cinema, he always tries to bring in something very different. I was really impressed by his last attempt in 2001 for Aks (Amitabh Bachchan and Manoj Bajpai ), and he definitely has succeeded in making another fabulous master piece. The highlight of the movie is the great use of cinematography techniques. Its probably for the first time in Indian cinema that juxtaposing has been used with such a great effect. The way each character gradually immerses into scenes from the past leaves you fantasizing about the hard work that has gone in the background to create this, both technically and on the part of the actor.

The movie is correct representation of the indian corruption situation intensifying the claims made by the common people. It has shown the power and will of the people. Moreover it should be noted that the deaths due to the MiG places was a reality and was considered one of the most gruesome way of death, since the men and women of the country who joined the arm forces, do so to serve the nation, not so that people in power can skim off the top and give the planes rusty parts which might cause deaths in weirdest manners possible. 

\section*{Swades}


Thank god for swades. Its been ever so often that as Indians we have lamblasted the mushy, one track nonsensical movies that were being churned out by bollywood and yearned for something slightly honest and celebral. And now it seems that the moment has arrived. Swades has finally reached the next level of Indian movie-making with the help of ashutosh, srk and rahman where the execution, the concept and the treatment of the film is at least (i can safely assume) close to world standards. There is no point comparing lagaan (the directors 1st movie) to this movie as his earlier venture was purely fiction with no semblance of realism in it. However this movie has so many ingredients of being a pathbreaker for Indian cinema that its almost scary. The feelings and expectations of an nri set aside, it tells the story of middle and upper class India losing touch with the harsh realities of the sufferings of the less fortunate ones all around us (and the fact that we can do very little to change the ongoing scenario is even more glaring and painful). Besides this the movie also deals with the desires and priorities of humans inspite of the trials and tribulations of daily living.

The maturity of the script, the actors and the score makes it a journey of epic proportions without it looking epic at all. The normalcy of the movie is something which might (and hopefully will) change the way cinema is seen and made in India in the future. One can only hope.

A jaw drapping experience of a man who comes to visit his mother and has a life changing experience.
The movie was directed by Ashutosh Gowarikar.
\end{document}

\documentclass{article}
\usepackage[utf8]{inputenc}
\usepackage{geometry}
\geometry{
 a4paper,
 total={170mm,257mm},
 left=20mm,
 top=20mm,
 }


\title{Movie review}
\author{Akash, Ankit, Prannay}
\date{April 2017}

\begin{document}

\maketitle

\section*{Rang De Basanti}
A fantastic movie of Mr Perfectionist.
The director of the movie is Rakesh Om Prakash Mehra.
Paint It Yellow…. Oops….that's suppose to be Rang De Basanti….. Well that's exactly the director tries to convey. Its about today, us and our present, yet the similarities we have from the Pre-independence era. The Gen-x who knows Mac-D but still prefers the Dhaba Paranthas with sweet Lassi. However they restrict their national values only to food and nothing more. Its not a run of the mill stuff with 6 six romantic songs, couple of foreign locales and then finally some emotional drama….. No no no….. Rang De…is a Cult movie. It is more of an introspection, a food for our thought process. It makes us think, as to how should we actually celebrate our freedom. I was really moved by some of the ending lines by the narrator, "I thought there are 2 types of people in the world, one who die crying and other who move away in silence but today I learned there is third genre, people who go laughing". That says it all… 

Rang De… is definitely a very brave and innovative attempt by the director and for that matter he has selected a near perfect cast. However veterans like Om Puri and Anupam Kher looked disposed. All the characters grow gradually in the film and make you think their way. Everyone is given enough space to justify their talent. Rakesh Omprakash Mehra is a director of the new emerging Indian cinema, he always tries to bring in something very different. I was really impressed by his last attempt in 2001 for Aks (Amitabh Bachchan and Manoj Bajpai ), and he definitely has succeeded in making another fabulous master piece. The highlight of the movie is the great use of cinematography techniques. Its probably for the first time in Indian cinema that juxtaposing has been used with such a great effect. The way each character gradually immerses into scenes from the past leaves you fantasizing about the hard work that has gone in the background to create this, both technically and on the part of the actor.


\section*{Swades}
A jaw drapping experience of a man who comes to visit his mother and has a life changing experience.
The movie was directed by Ashutosh Gowarikar.
\end{document}
